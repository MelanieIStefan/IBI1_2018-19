\documentclass[pdflatex,a4paper]{article}

\usepackage{pgffor, ifthen}
\newcommand{\notes}[3][\empty]{%
    \noindent \vspace{10pt}\\
    \foreach \n in {1,...,#2}{%
        \ifthenelse{\equal{#1}{\empty}}
            {\rule{#3}{0.5pt}\\}
            {\rule{#3}{0.5pt}\vspace{#1}\\}
        }
}

\usepackage{graphicx}

\usepackage[margin=3cm]{geometry}

\title{Practical 12. Modelling infections}

\author{MI Stefan}

\date{IBI1, 2018/19}


\usepackage{url}
\usepackage{amsmath}

\begin{document}

\newcommand{\<}{\textless}
\renewcommand{\>}{\textgreater}


\maketitle

\section{Learning objectives}

\begin{itemize}
\item

\end{itemize}

\section{Introduction}

SIR models (\textbf{S}usceptible - \textbf{I}nfected - \textbf{R}ecovered) are used to study how infectious diseases spread through a population. In its simplest form, the model assumes that the population being studied falls into three separate groups: Susceptible individuals (S) are healthy, but may contract the disease. Infected people (I) have contracted the disease and can pass it on to susceptible people at a rate that depends on an infection probability upon contact \(\beta\) and on the proportion of infected people in the population. An infected person can recover (R) with recovery probability \(\gamma\), and is assumed to be immune to the disease after recovery.  

Though simple, SIR models can be very informative. They can be extended and modified in many ways to account for specific diseases, interventions, incubation periods, or other factors. 

SIR models can be deterministic or probabilistic - in this practical, we will look at four different probabilistic models. We will build the models step-by-step, using tools you have encountered before or that are supplied in this practical guide. As often in programming, there are more than one correct ways to do something, so if you have a good idea, feel free to use it!


\section{A simple SIR model}

\begin{itemize}
\item
Make a new python script called \verb=SIR.py=
\item
At the beginning, you have to import the relevant python libraries in order to have access to some of the functions we need. Start with that.

\begin{lstlistings}

\end{lstlistings}
\end{itemize}



\section{For your portfolio}

The markers will look for and assess the following:

\begin{description}
\item[File variables.py] $\;$\\
\begin{itemize}
\item

\end{itemize}




\end{description}



You can add or edit things after the Practical session. We do not look at the commit date, we just want it all to be there!

\end{document}


